
% -------------------
% Imprimindo a capa
% -------------------
\imprimircapa


% -------------------
% Imprimindo a folha de rosto
% -------------------

% \imprimirfolhaderosto % adicionar * para indicar ficha bibliográfica

% % -------------------
% % Dedicatória
% % -------------------

% \begin{dedicatoria}

%  Dedico este trabalho a todos aqueles que apesar das condições ideológicas e econômicas de nosso país, nunca deixaram de acreditar que através da ciência podemos construir um mundo melhor para todos.
 
%  Dedico este trabalho e a conclusão do curso, para Inete Rosa Rios, minha querida avó! Você vai estar eternamente na minha memória, jamais esquecerei o que fizestes por mim.

% \end{dedicatoria}

% % ------------------
% % Agradecimentos
% % ------------------

% \begin{agradecimentos}

% Gostaria de agradecer esta conquista ao fundador do Universo. Ele e somente ele sabe do milagre de transformação da minha vida, que me permitiu subir mais esse degrau. A minha família, em especial, minha avó Inete Rosa Rios, porque sem ela eu não teria condições de sobrevivência e não teria aprendido sobre os valores da família e mais importante amar ao próximo.

% Gostaria de agradecer ao Dr. Rogério Corrêa, que acreditou na minha capacidade, me convidando para participar desse projeto multidisciplinar e teve paciência me iluminando sempre que me faltou conhecimento. Gostaria de agradecer à UNIVALI, por me proporcionar este tempo de aprendizado incrível, com excelentes professores, possibilitando a compreensão do que é o mundo lá fora, sendo de fato, uma universidade que me abriu portas para o universo.

% Gostaria de agradecer a Dra. Gizelle Inácio Almerindo, que sempre disposta me ajudando, dentro e fora de sala de aula, que me instruiu nesse tempo de investigação científica, principalmente sobre os estudos cinéticos. Que sua luz continue transformando professora.

% Quero agradecer aos meus amigos Johann V. Hemmer, Lucas H. da Costa, Guilherme G. Motta, Kaio P. Eifler, Mateus R. Goulart, Eduardo Steffens, Mateus Siqueira e todos aqueles outros que me ajudaram com conhecimento ou acreditando no meu potencial. Sem vocês minha jornada teria sido muito mais difícil.

% Agradeço ao laboratório de Síntese Orgânica da Univali, situado no Bloco F6, sala 212, pela disponibilidade dos reagentes e pela equipe que o mantém. Agradeço a equipe técnica da Central Laboratorial de Ensaios Analíticos - Clean, que disponibilizou espaço para realização de meus procedimentos e que me acolheu como colega de trabalho, possibilitando o desenvolvimento de desta iniciação científica.

% \end{agradecimentos}

% % -------------------
% % Epígrafe
% % -------------------

% \begin{epigrafe}
% \vspace*{\fill}
%     \begin{flushright}
%         \begin{minipage}{.5\textwidth}
%             \begin{flushright}
%            \textit{“Que nada lhe faltava para reinar, \\
%         exceto o reino. - Nicolau Maquiavel"}
%             \end{flushright}
%         \end{minipage}
%     \end{flushright}
% \end{epigrafe}

% % -------------------
% % Resumo em língua vernácula
% % -------------------

\begin{resumo}

Texto do resumo

\noindent
\textbf{Palavras-chave}: Compostos chalcônicos. Gênese planejada. Docking molecular. Reator \textit{CSTR}.

\end{resumo}

% % -------------------
% % Resumo em língua estrangeira
% % -------------------

% % \newpage
% % \begin{resumo}[ABSTRACT]

% % The goal of this work is to present, in simple manner, the correct utilization of the univali package, and the \abnTeX \ class for the elaboration of scientific and academic papers in the format suggested by ABNT and also deemed correct by UNIVALI's faculty.

% % \noindent
% % \textbf{Keywords}: \abnTeX. UNIVALI. Papers.
% % \end{resumo}
% \newpage

% % -------------------
% % Lista de ilustrações
% % -------------------

% \pdfbookmark[0]{\listfigurename}{lof}
% \listoffigures
% \cleardoublepage

% % -------------------
% % Lista de tabelas
% % -------------------

% \pdfbookmark[0]{\listtablename}{lot}
% \listoftables
% \cleardoublepage

% % ---
% % Lista de siglas
% % ---

\begin{siglas}
    \item [MW] Massa molecular
\end{siglas}

% % ---
% % Lista de símbolos
% % ---

% \begin{simbolos}
%     \item[$\alpha$] Alfa
%     \item[$\beta$] Beta
%     \item[$\Delta \phi$] Atividade biológica
%     \item[\AA] Ângstrom
%     \item[$\log{P}$] Lipofilicidade
%     \item[$\pi$] Constante de hidrofobicidade
%     \item[$\sigma$] Constante de Hammet
%     \item[$\rho$] Constante rô
%     \item[$\sigma$] Fator de perda
%     \item[$\mu$] Viscosidade
% \end{simbolos}

% % -------------------
% % Sumário
% % -------------------
% \pdfbookmark[0]{\contentsname}{toc}
% \tableofcontents*
% \cleardoublepage